\documentclass[12pt, a4paper, simple]{eskdtext}

\usepackage{hyperref}
\usepackage{env}
\usepackage{_sty/gpi_lst}
\usepackage{_sty/gpi_toc}
\usepackage{_sty/gpi_t}
\usepackage{_sty/gpi_p}
\usepackage{_sty/gpi_u}

% Код
% \ESKDletter{О}{Л}{Р}
% \def \gpiDocTypeNum {81}
% \def \gpiDocVer {00}
% \def \gpiCode {\ESKDtheLetterI\ESKDtheLetterII\ESKDtheLetterIII.\gpiStudentGroupName\gpiStudentGroupNum.\gpiStudentCard-0\gpiDocNum~\gpiDocTypeNum~\gpiDocVer}

\def \gpiDocTopic {Отчёт лабораторной работы №\gpiDocNum}


% колонтитулы
\usepackage{fancybox, fancyhdr}
\fancypagestyle{plain}
{
    \renewcommand{\footrulewidth}{0pt}          % Толщина отделяющей полоски снизу
    \renewcommand{\headrulewidth}{0pt}          % Толщина отделяющей полоски сверху
    % \fancyhead[C]{\hfill\gpiCode\hfill}         % Коллонтитул сверху
    \fancyfoot[C]{\hfillстр.~\thepage}          % Коллонтитул снизу
}

% Графа 1 (наименование изделия/документа)
% \ESKDcolumnI {\ESKDfontII \gpiTopic \\ \gpiDocTopic}

% Графа 2 (обозначение документа)
% \ESKDsignature {\gpiCode}

% Графа 9 (наименование или различительный индекс предприятия) задает команда
% \ESKDcolumnIX {\gpiDepartment}

% Графа 11 (фамилии лиц, подписывающих документ) задают команды
% \ESKDcolumnXIfI {\gpiStudentSurname}
% \ESKDcolumnXIfII {\gpiTeacherSurname}
% \ESKDcolumnXIfV {\gpiTeacherSurname}

\begin{document}
    \input{_tex/gpi_rep_titlePage.tex}
    \ESKDstyle{empty}
    \thispagestyle{plain}
    \pagestyle{plain}

    % \paragraph{}
    \begin{center}
        \textbf{\gpiDocTopic}
    \end{center}

    % = = = = = = = =
    \paragraph{} \textbf{Тема}: <<\gpiTopicRep>>

    \paragraph{} \textbf{Цель}: 1) создать набор жестов; 2) использовать созданные жесты в приложении.

    \paragraph{} \textbf{Что нужно сделать}:
    Скачать приложение <<Gesture Builder Tool>>.
    Нарисовать жесты <<1>>, <<2>>, <<S>>.
    Созранить жесты под именем <<1>>, <<2>>, <<S>>.
    Найти файл жестов на карте устройства и перекинуть его в Android Studio. 
    Наследовать интерфейс OnGesturePerformedListener.
    Реализовать метод onGesturePerformed, который вызывается при распозновании жеста.
    Вывести на экран вероятности жеста, либо сообщить о его отсутствии.

    \paragraph{} \textbf{Ход работы}:

    Скачиваем приложение <<Gesture Builder Tool>> из Play маркета.
    Скриншот на рисунке~\ref{fig:PlayStore_GestureBuilderTool}.

    Создаем новый жест (жест <<1>>). Скриншот на рисунке~\ref{fig:GestureBuilderTool_create_1}.

    Рисуем жест (жест <<1>>). Скриншот на рисунке~\ref{fig:GestureBuilderTool_draw_1}.

    Сохраняем жест (жест <<1>>). Скриншот на рисунке~\ref{fig:GestureBuilderTool_save_1}.

    Создаем новый жест (жест <<2>>). Скриншот на рисунке~\ref{fig:GestureBuilderTool_create_2}.

    Рисуем жест (жест <<2>>). Скриншот на рисунке~\ref{fig:GestureBuilderTool_draw_2}.

    Сохраняем жест (жест <<2>>). Скриншот на рисунке~\ref{fig:GestureBuilderTool_save_2}.

    Создаем новый жест (жест <<S>>). Скриншот на рисунке~\ref{fig:GestureBuilderTool_create_S}.

    Рисуем жест (жест <<S>>). Скриншот на рисунке~\ref{fig:GestureBuilderTool_draw_S}.

    Сохраняем жест (жест <<S>>). Скриншот на рисунке~\ref{fig:GestureBuilderTool_save_S}.

    Смотрим куда сохранились жесты, нажав на кнопку <<?>>. Скриншот на рисунке~\ref{fig:GestureBuilderTool_where_saved}.

    Находим файл жестов на карте Android. Скриншот на рисунке~\ref{fig:gestures_txt}.

    \begin{figure}[!h]
        \centering
        \begin{minipage}{0.24\textwidth}
            \centering
            \includegraphics[height=6cm]
                {_assets/PlayStore_GestureBuilderTool.jpg}
            \caption{(1) Скачиваем в маркете}
            \label{fig:PlayStore_GestureBuilderTool}
        \end{minipage}
        \begin{minipage}{0.24\textwidth}
            \centering
            \includegraphics[height=6cm]
                {_assets/GestureBuilderTool_create_1.jpg}
            \caption{(2) Создать жест}
            \label{fig:GestureBuilderTool_create_1}
        \end{minipage}
        \begin{minipage}{0.24\textwidth}
            \centering
            \includegraphics[height=6cm]
                {_assets/GestureBuilderTool_draw_1.jpg}
            \caption{(3) Рисуем жест}
            \label{fig:GestureBuilderTool_draw_1}
        \end{minipage}
        \begin{minipage}{0.24\textwidth}
            \centering
            \includegraphics[height=6cm]
                {_assets/GestureBuilderTool_save_1.jpg}
            \caption{(4) Сохраняем жест}
            \label{fig:GestureBuilderTool_save_1}
        \end{minipage}
    \end{figure}

    \begin{figure}[!h]
        \centering
        \begin{minipage}{0.24\textwidth}
            \centering
            \includegraphics[height=6cm]
                {_assets/GestureBuilderTool_where_saved.jpg}
            \caption{(11) Куда сохранен файл жестов}
            \label{fig:GestureBuilderTool_where_saved}
        \end{minipage}
        \begin{minipage}{0.24\textwidth}
            \centering
            \includegraphics[height=6cm]
                {_assets/GestureBuilderTool_create_2.jpg}
            \caption{(5) Создать жест}
            \label{fig:GestureBuilderTool_create_2}
        \end{minipage}
        \begin{minipage}{0.24\textwidth}
            \centering
            \includegraphics[height=6cm]
                {_assets/GestureBuilderTool_draw_2.jpg}
            \caption{(6) Рисуем жест}
            \label{fig:GestureBuilderTool_draw_2}
        \end{minipage}
        \begin{minipage}{0.24\textwidth}
            \centering
            \includegraphics[height=6cm]
                {_assets/GestureBuilderTool_save_2.jpg}
            \caption{(7) Сохраняем жест}
            \label{fig:GestureBuilderTool_save_2}
        \end{minipage}
    \end{figure}

    \begin{figure}[!h]
        \centering
        \begin{minipage}{0.24\textwidth}
            \centering
            \includegraphics[height=6cm]
                {_assets/gestures.txt.jpg}
            \caption{(12) Файл с жестами на карте}
            \label{fig:gestures_txt}
        \end{minipage}
        \begin{minipage}{0.24\textwidth}
            \centering
            \includegraphics[height=6cm]
                {_assets/GestureBuilderTool_create_S.jpg}
            \caption{(8) Создать жест}
            \label{fig:GestureBuilderTool_create_S}
        \end{minipage}
        \begin{minipage}{0.24\textwidth}
            \centering
            \includegraphics[height=6cm]
                {_assets/GestureBuilderTool_draw_S.jpg}
            \caption{(9) Рисуем жест}
            \label{fig:GestureBuilderTool_draw_S}
        \end{minipage}
        \begin{minipage}{0.24\textwidth}
            \centering
            \includegraphics[height=6cm]
                {_assets/GestureBuilderTool_save_S.jpg}
            \caption{(10) Сохраняем жест}
            \label{fig:GestureBuilderTool_save_S}
        \end{minipage}
    \end{figure}

    \newpage

 % \paragraph{} \textbf{Разработка дизайна}:

    % \begin{figure}[h!]
    %     \centering
    %     \includegraphics[width=12cm]
    %         {_assets/PlayStore_GestureBuilderTool.jpg}
    %     \caption{onFling}
    % \end{figure}

    \paragraph{} \textbf{Разработка дизайна}:

    \begin{figure}[!h]
        \centering
        \begin{minipage}{0.24\textwidth}
            \centering
            \includegraphics[height=6cm]
                {_assets/i_draw_1.jpg}
            \caption{Жест <<1>> - распознан}
            \label{fig:i_draw_1}
        \end{minipage}
        \begin{minipage}{0.24\textwidth}
            \centering
            \includegraphics[height=6cm]
                {_assets/i_draw_2.jpg}
            \caption{Жест <<2>> - распознан}
            \label{fig:i_draw_2}
        \end{minipage}
        \begin{minipage}{0.24\textwidth}
            \centering
            \includegraphics[height=6cm]
                {_assets/i_draw_3.jpg}
            \caption{Жест <<3>> - не существует}
            \label{fig:i_draw_3}
        \end{minipage}
        \begin{minipage}{0.24\textwidth}
            \centering
            \includegraphics[height=6cm]
                {_assets/i_draw_S.jpg}
            \caption{Жест <<S>> - распознан}
            \label{fig:i_draw_S}
        \end{minipage}
    \end{figure}

    \newpage
    
    \paragraph{} \textbf{Исходный код}: 

    \lstinputlisting[language=xml, name=app/src/main/res/values/strings.xml]
        {../gpi_src/gpi_rpodms_lab8/app/src/main/res/values/strings.xml}

    \lstinputlisting[language=xml, name=app/src/main/res/layout/activity_main.xml]
        {../gpi_src/gpi_rpodms_lab8/app/src/main/res/layout/activity_main.xml}

    \lstinputlisting[language=java, name=app/src/main/java/.../MainActivity.java]
        {../gpi_src/gpi_rpodms_lab8/app/src/main/java/io/github/Pavel_Innokentevich_Galanin/gpi_rpodms_lab8/MainActivity.java}

    \paragraph{} \textbf{Вывод}:
    Скачали приложение <<Gesture Builder Tool>> из Play маркета.
    Нарисовали жесты <<1>>, <<2>>, <<S>>.
    Созранить жесты под именем <<1>>, <<2>>, <<S>>.
    Нашли файл жестов на карте устройства и перекинули его в Android Studio. 
    Наследовали интерфейс OnGesturePerformedListener.
    Реализовали метод onGesturePerformed, который вызывается при распозновании жеста.
    Вывели на экран вероятности жеста, либо сообщили о его отсутствии.

    % = = = = = = = =
    \paragraph{} \textbf{Список использованных источников} 
    % \addcontentsline{toc}{section}{СПИСОК ИСПОЛЬЗОВАННЫХ ИСТОЧНИКОВ}
    % \section*{Список использованных источников}
    \begin{enumerate}
        \item[1.] Кондратюк, А.П. Разработка приложений для мобильных операционных систем «Android» :
        ЭУМК для студ. второй ступени (магистратуры) специальности 1-31 81 06 <<Веб-программирование и интернет-технологии>>
        физ.-мат. фак. / А.П. Кондратюк ; Брест. гос. ун-т им. А.С. Пушкина, каф. ПМиИ. – Брест :
        электрон. издание БрГУ, 2016. – 469 с.\\
        §23. Лабораторная работа №6, cc. 401-414.
    \end{enumerate}
    \newpage
\end{document}
